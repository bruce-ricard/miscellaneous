\documentclass{beamer}
\usepackage{graphicx}
\usepackage{amsmath}
\begin{document}
\title{Public Key Infrastructure}
\subtitle{Introduction to cryptography}
\author{Bruce Ricard, VMware}
\date{April 30th, 2020}

\frame{\titlepage}

\frame{\
  \frametitle{Agenda}
  \tableofcontents
}

\section{Prerequisites}
\subsection{Cryptography}
\frame{
  \frametitle{Cryptography}
  \framesubtitle{A little history}

  Caesar cipher
  \pause
  \begin{figure}
  \includegraphics[scale=0.3]{images/caesar_cipher.png}
  \end{figure}

  Example: $``I LOVE CRYPTOGRAPHY'' \rightarrow
  ``L ORYH FUBSWRJUDSKB''$

  \pause

  \vspace{\baselineskip}
  Symmetric(-Key) Cryptography
}

\frame{
  \frametitle{Cryptography}
  \framesubtitle{Vocabulary}

  Symmetric Cryptography diagram:

  plain text $\xrightarrow[key]{encryption}$
  cipher $\xrightarrow[key]{decryption}$ plain text
}

\frame{
  \frametitle{Cryptography}
  \framesubtitle{Symmetric cryptosystem examples}

  \begin{itemize}
  \item Caesar cipher
  \item Data Encryption Standard (DES) (1975)
  \item Advanced Encryption Standard (AES) (1998)
  \end{itemize}
}


\frame{
  \frametitle{Cryptography}
  \framesubtitle{Issues with symmetric cryptography}

  \begin{itemize}
  \item Sharing key with other parties
  \item Different key for every party
  \end{itemize}

  $\rightarrow$ Public Key cryptography
}

\frame{
  \frametitle{Cryptography}
  \framesubtitle{Public key cryptography}

  One way functions
  \pause

  In $\mathbb{Z}/100\mathbb{Z}$ (numbers modulo 100)

  $$ f : n \mapsto n^2 $$

  $f(41) = 41^2 = 1681 = 81 \pmod{100}$

}

\frame{
  \frametitle{Cryptography}
  \framesubtitle{Public key cryptography}

  RSA (Rivest–Shamir–Adleman 1977)
  \begin{itemize}
  \item Public Key: modulo $n$, number $e$
  \item Private Key: number $d$
  \end{itemize}
  Such as for any $x$, number modulo $n$:
  $$ x^{e.d} = x \pmod{n} $$

  \pause

  Encryption: $ p \mapsto p^e $ ;
  Decryption: $ c \mapsto c^d $
}

\frame{
  \frametitle{Cryptography}
  \framesubtitle{Issues with public key cryptography}

  Slow, can only encrypt little data
  (by design -- one-way functions)

  Anybody can encrypt data: no more sender validation

  \vspace{\baselineskip}
  $\rightarrow$ Digital Signatures
}

\frame{
  \frametitle{Cryptography}
  \framesubtitle{Signing / Validating}

  Use Public Key Cryptography to sign: but ``encrypt'' with private
  key.

  \pause

  \begin{itemize}
  \item Public Key: modulo $n$, number $e$
  \item Private Key: number $d$
  \end{itemize}
  Such as for any $x$, number modulo $n$:
  $$ x^{e.d} = x \pmod{n} $$

  Encryption: $ p \mapsto p^e $ ;
  Decryption: $ c \mapsto c^d $

  \pause

  Signing: $ p \mapsto p^d $ ;
  Validation: $ s \mapsto s^e $
}

\frame{
  \frametitle{Cryptography}
  \framesubtitle{Issues with signatures}

  Signing proves that you own the Key Pair.

  \pause
  \vspace{\baselineskip}

  But how do you prove who you actually are?

  \pause
  \vspace{\baselineskip}

  $\rightarrow$ Certificates
}


\frame{
  \frametitle{Cryptography}

  Questions?
}

\section{Certificates}
\frame{
  \frametitle{ID cards}
  \framesubtitle{why do we use them?}

  Why do we have IDs?

  \pause

  Proof for information

  \pause

  The 3 main categories on an ID card?

  \pause

  \begin{itemize}
  \item Proof that it belongs to a certain entity \& method for
    anybody to validate that
  \item Proof that it was emitted by a trusted authority
  \item Data the entity wants certified
  \end{itemize}

  \pause

  Authorities in place:
  \begin{itemize}
  \item State government: Certificate Authority (CA)
  \item DMV: Registration Authority (RA)
  \item TSA agent with ultraviolet light: Validation Authority (VA)
  \end{itemize}
}

\frame{
  \frametitle{Certificates}
  \framesubtitle{identity}

  electronic-IDs for software \pause

  \begin{itemize}
  \item CA
  \item RA
  \item x509 \pause
    \begin{itemize}
    \item Name / URL
    \item public key
    \item trusted signature
    \end{itemize}
  \end{itemize}
}

\frame{
  \frametitle{Certificates}

  Questions?
}

\section{Communicating safely using cryptography}
\frame{
  \frametitle{Using cryptography to communicate safely}

  \begin{itemize}
  \item Encrypt data so that it can only be read by entity with decoding key
  \item Making sure that the right entity has the decoding key
  \end{itemize}

  \pause
  \begin{itemize}
  \item Encrypt AES key with RSA
  \item Sign RSA public keys / owner by trusted party \pause : certificates
  \end{itemize}
}


\subsection{SSL / TLS}
\frame{
  \frametitle{SSL / TLS}
  \framesubtitle{TLS - Transport Layer Security}
  \begin{itemize}
  \item Cryptographic protocol for safe communication over computer network
    \pause
  \item Previously: SSL - Secure Sockets Layer (1994)
    \pause
  \item TLS 1.0 (1999)
  \item Now TLS 1.3 (2018)
  \end{itemize}
}


\frame{
  \frametitle{SSL / TLS}
  \framesubtitle{TLS handshake (simplified)}

  \pause
  \begin{enumerate}
  \item Client makes $https$ call to server: initiates TLS handshake
  \item Server sends its $x509$ certificate
  \item Client validates certificate signature against CA
  \item Client validates certificate hostname and gets public key $p$
  \item Client generates random string to be symetric private key $k$
  \item Client encrypt $k$ with server's public key $p$, and sends it to the server
  \item Server decrypts encoded key with its private key
  \end{enumerate}

  \pause
  End of handshake:
  \begin{itemize}
  \item client validated identity of server
  \item both parties have secure secret key for encrypted private communication
  \end{itemize}

  \pause
  Server and client will now communicate by encoding and decoding any data sent
  over the wire with their private key, using a symetric algorithm like AES.
}

\frame{
  \frametitle{How does TLS work?}
  \framesubtitle{Why is it secure?}

  \begin{itemize}
  \item encrypts data
  \item makes sure the right entity can decrypt
  \end{itemize}
}

\frame{
  \frametitle{TLS}

  Questions?
}

\frame{
  \frametitle{TLS}
  \framesubtitle{What is --skip-ssl-validation?}

  Or in the browser ``This page in not secure...''
  \pause

  Not encrypting data? \pause Yes it is.
  Not validating identity.
}

\frame{
  \frametitle{mTLS}
  \framesubtitle{mutual TLS}

  \begin{itemize}
  \item With TLS, client validates identity of server. Client doesn't authenticate.
  \item Most websites use username/password for client authentication.
  \item mTLS can be done with username/password: we'll look at certificate based mTLS
  \end{itemize}

  \pause
  Not MTLS
}

\frame{
  \frametitle{mTLS}
  \framesubtitle{mutual TLS}

  \begin{enumerate}
  \item server sends its certificate to client
  \item client validates certificate signature and hostname
  \item client sends its certificate to server
  \item server validates certificate signature
  \item server generates random key to use as private symmetric key and sends
    to client
  \end{enumerate}

  \pause

  Both entities validated that the other party is trustworthy
}

\frame{
  \frametitle{mTLS}
  \framesubtitle{TLS termination}

  \begin{itemize}
  \item TLS termination proxy
  \item TLS forward proxy
  \end{itemize}
  \pause
  CF does mostly TLS forward proxy, calls it TLS termination proxy
}


\section{PKI}
\subsection{What is a public key infrastructure?}
\frame{
  \frametitle{PKI}

  A PKI consists of:
  \begin{itemize}
  \item A certificate authority (CA)
  \item A registration authority (RA)
  \item A central directory (for key storage)
  \item A certificate management system (e.g. who can access the stored keys)
  \item A certificate policy (stating the PKI's requirements concerning its procedures.
    Its purpose is to allow outsiders to analyze the PKI's trustworthiness.)
  \end{itemize}
}

\section{Extras}

\frame{
  \frametitle{Safe key exchange}
  \framesubtitle{Diffie Helmann}

  \begin{itemize}
  \item Issue with exchanging private keys with symmetric algorithm
    if private key gets lost
  \item Diffie Helmann algorithm
  \end{itemize}
}
\end{document}
